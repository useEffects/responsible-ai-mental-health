\section{Introduction}
Artificial Intelligence (AI) is becoming deeply integrated into daily life, profoundly transforming communication and interactions. Over 50\% of Internet traffic is generated by machine-to-machine communication, with person-to-machine interactions also increasing. Chatbots, softbots, and virtual assistants are becoming essential in everyday life \cite{elliott2019culture}. Recent advances in generative AI, such as ChatGPT and Stable Diffusion, are approaching human-level capabilities in intelligence \cite{orru2023human}. Large language models (LLMs) are a type of artificial intelligence designed to understand, generate, and manipulate human language. With the integration of text-to-image models like Stable Diffusion, these virtual assistants can now visually represent themselves as human avatars or pets, enhancing the interaction experience with both textual and visual elements. \\
Loneliness is defined as an unpleasant emotional response to perceived isolation, often characterized by a lack of meaningful social connections and intimacy \cite{lonelinessWiki}. Loneliness is increasingly recognized as the next critical public health issue. A plausible reason for this concern may be related to emerging societal trends affecting the way we relate, communicate, and function in our social environment. If we compare two individuals of the same age - one today and another one a generation ago - we would find that the one today is more likely to feel lonely. This is based on the idea that there have been societal changes - such as the rise of living alone - that make newer generations more likely to feel lonely \cite{owid-loneliness-epidemic}. In England, the Office for National Statistics conducts the Community Life Survey, in which they ask people how often they feel lonely. According to this data, those aged 16 to 24 are the group most likely to report feeling lonely, with 10\% feeling lonely "often or always". In contrast, those aged 65 years and older are the group least likely to report feeling lonely, with 3\% feeling lonely "often or always". \\
In the realm of mental health, LLMs offer promising solutions for addressing loneliness. By providing empathetic conversations and companionship through virtual avatars. Recent studies indicate that AI companions can effectively mitigate loneliness by providing emotional support and facilitating meaningful interactions. For instance, findings suggest that these systems can evoke feelings of being heard and understood, which are critical components in reducing loneliness \cite{strohmann2023toward} and beneficial for individuals seeking companionship \cite{odekerken2020mitigating}. People suffering from depression can sometimes find it difficult to leave the house to get medical assessment, or to go see a therapist. People with autism can find interaction with other humans very difficult, especially with people they don't know. For instance, children with autism could use videos generated by the AI in order to acquire certain competences, and then test such competences in the real world once they feel confident and ready.
However, the deployment of such technologies raises important ethical considerations. As AI companions become more integrated into daily life, it is essential to ensure their design and use are grounded in ethical principles that prioritize user well-being and safeguard against potential misuse. Privacy and ethical concerns remain with using generative AI for tackling loneliness, such as the potential for AI companions to replace human relationships, the risk of users becoming overly dependent on AI, and the need to ensure AI systems are transparent, accountable, and aligned with human values

