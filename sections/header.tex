\copyrightyear{2024}
\copyrightclause{Copyright for this paper by its author.
    Use permitted under Creative Commons License Attribution 4.0
    International (CC BY 4.0).}

\conference{1st International Workshop on Responsible AI for Healthcare and Net Zero (Hybrid),
    October 16--17, 2024, IIT Madras, Chennai, India.}

\title{AI-Driven Companionship in Addressing Loneliness: Ensuring Ethical and Responsible Use of LLMs}

\author{Joel Samuel Raj Amose}[
    orcid=0009-0006-0131-2892,
    email=joel.20fit18@gct.ac.in,
    url=https://useEffects.github.io/
]

\begin{abstract}
    Artificial Intelligence (AI) is gradually penetrating into people's lives and changing the ways of communication and interactions significantly. Currently, more than 50\% of all Internet traffic is made by the exchange of information between machines, rather than persons and machines. The interactions between people and machines are also increasing. The concepts of chatbots, softbots, and virtual assistants become vital in people's daily lives. The newly released generative AI tools like the ChatGPT and the Stable Diffusion are now nearly at par with human capabilities. Thus, this paper seeks to discuss the prospects and issues with the use of artificial intelligence in the companionship addressing loneliness. The study critically examines the issues of ethicality of using artificial intelligence companions, which is the potential of developing fake connection, fostering negative practices, and making the public aware of AI potential. The major focus lies with AI usage; need for AI to complement real-world social contacts; and the need to protect users' security and privacy and autonomy. The paper argues for a cautious and responsible approach to AI-driven companionship, advocating for interdisciplinary collaboration to address these ethical concerns and design solutions that align with human needs for meaningful interaction.
\end{abstract}

\begin{keywords}
    Artificial Intelligence \sep
    Healthcare \sep
    Positive Mental Health \sep
    Ethics \sep
    Responsible AI \sep
    AI Companionship \sep
    Digital Interaction \sep
    Emotional Support
\end{keywords}

\maketitle
