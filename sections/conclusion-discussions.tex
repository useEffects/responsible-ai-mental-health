\section{Discussions and Conclusion}
To summed up, artificial intelligence in the form of virtual companions supported by the progress of the large language models (LLMs) provide a very appropriate solution that can be used in combating increasing occurrence of loneliness in society of the present day. AI companions should not, in the course of the application, strengthen negative feelings or encourage poisonous ideas. Given the influence that LLMs can pose to the vulnerable users especially those who are suffering from depression or anxiety, ethical policies should prohibit the posting of one's material that will be injurious to others or the public. That's why, it is important to constantly adjust the AI models, providing them with supportive and positive behavior guidance, and excluding any negative impact on the user. 
Another rather acute question is consent and transparency. Users should be fully being aware of the type of AI that they are dealing with including the capability of AI in emotion. Competence and willingness to bring out the intent to care. More importantly, it is mandatory that users do not confuse with the interfaces that are AI-driven.Acompaniment for real human affection as this may lead to struggle in sorting emotions or even disappointment. Moreover, it entails protecting users' information and their privacy as well when deploying artificial intelligence. AI subordinate often count on sensitive personal information to perform their work which increases data security, misuse, possible breaches of confidentiality and. The ethical frameworks have to put in place strict measures to ensure that users' information provided are not exploited by third parties. The features of the AI companions can bring improvements to the  feeling of belonging they are still devoid of ability to give and recognize social status of others and therefore the aspect of empathy.  Presented fears of engendering a number of shallow social contacts which might in fact increase loneliness levels in the long run. Finally this branch needs further continual interdisciplinary study to assess the future psychological  impacts of AI companionship. Research should seek to find out if LLMs can actually reduce loneliness or if they only offer a sort of 'band-aid' solution potentially exacerbating the degrees of social exclusion among its recipients. Additionally,policymakers should participate in these discussion so that they are in a position to regulate the kind of companionship that AI brings. That is conductive to human dignity, respect as well as their well-being.